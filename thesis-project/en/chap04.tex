\chapter{Scripter's Documentation}

While we cannot change parts of the engine without recompilation of its source code, we can provide implementation of various parts of the
game through scripts written in the Lua programming languagei\footnote{To learn about the language, we recommend the Programming in Lua
book~\cite{PIL}.}. These scripts are mainly used during initialization and for the definition of
entities, enemy waves, research nodes and spells. In this section we will examine how to write these parts of the game.

A reference documentation for our Lua API can be found in the \texttt{lua-api} directory which is a part of the Attachment~B.

\section{Initialization}

When the game starts, it executes two script files -- \texttt{config.lua} and \texttt{init.lua}. The \texttt{config.lua} contains
a Lua table \texttt{config} with values used by the engine such as time periods and multipliers, blocks that are used by the default
level generator and a list of directories that contain scripts that are to be loaded when the game starts. The purpose of each value
is documented within this script file.

The file \texttt{init.lua} is used mainly to load other script files and user created mods. Additionally, it loads the values stored
in the configuration file to the engine. We can also use this file for any auxiliary commands we would like to be executed at the
start of the game -- e.g. for testing purposes.

If we want to modify our game we have two options. Our first option is to edit the already implemented Lua scripts that are located in the
\texttt{scripts} directory. This also includes creating new files and adding their names to the list of scripts that are to be loaded,
which can be found in the file \texttt{scripts/core.lua}. Our second option is to create a new scripting module, which is done by creating
a new directory in the game's root directory\footnote{The directory where the game's executable file is located.}
and adding this directory to our configuration script. This new directory has to contain a
script called \texttt{core.lua}, which will be executed by the game at startup. Note that scripting modules are executed precisely in
the order in which they are listed in the configuration script.

In Listing~\ref{core-lua-ex}, we can see an example of a small \texttt{core.lua} script located in the directory \texttt{some-mod}
in the game's root directory. It loads two additional scripts using the function \texttt{game.load}, which takes the path
to a scripts file which is relative to the game's root directory. Additionally, it redefines the function \texttt{game.init\_level},
which is called whenever the game creates a new empty level but before level generation starts. The important part of this function
is its return value -- if it is not \texttt{true}, the level generation will not happen. We can use this to write our own level
generator in Lua, which we would call from inside this function and then we would prevent the level generator written in \cpp by
returning \texttt{false}.

\begin{listing}[H]
    \centering
    \begin{lstlisting}
-- Scripts that will be loaded.
local scripts = {"script-1.lua", "script-2.lua"}

-- Path to this module from game root directory.
local path = "some-mod/"

-- Load the scripts, .. is used for string
-- concatenation.
for idx, script in ipairs(scripts) do
    game.load(path .. script)
end

-- Redefine the level init function.
game.init_level = function(width, height)
    game.print("A new level of some mod created!")
    game.print("Dimensions: " .. width .. ", " .. height)
    return true
end

-- We can execute any Lua code.
game.print("Some mod loaded!")
    \end{lstlisting}
    \caption{An example \texttt{core.lua} script.}
    \label{core-lua-ex}
\end{listing}

Scripts loaded in a \texttt{core.lua} script can execute any valid sequence of Lua commands and calls to our modding API, but they
are mainly used for the definition of new entities, enemy waves, research nodes and spells, which we will now examine.

\section{Entities}

To create a new entity, we need to create a Lua table that defines its components. In Listing~\ref{ent-table-ex1} we can see an example
of a simple ogre entity. Each entity definition needs to contain a table called \texttt{components}, which is a list of all component
types this entity has and is used by the game whenever it creates a new instance of this entity. For easier listing of components, the
file \texttt{scripts/enum.lua} -- which is loaded by default by the \texttt{scripts/core.lua} scripts -- contains an enum table called
\texttt{component} which contains variables representing the type identifiers of each component.

Once we list all of the components our new entity has, we need to define all of its components that require us to. In the file
\texttt{components.txt}, which is a part of the Attachment~B, we can find which components need to be defines in the entity definition
table and what fields do these component tables need to have. Note that the order in which we define component does not matter.

For this particular example entity, we firts define the \texttt{PhysicsComponent}, which represents the physical presence of an entity
inside of the game world. It has a single field called \texttt{solid}, which should be \texttt{true} for buildings that cannot be
walked through and \texttt{false} for any other entity. Then we define the \texttt{HealthComponent}, which represents the health of an
entity and its ability to be damaged and killed. It has three fields that need to be set -- \texttt{max\_hp}, which determines the
maximum amount of health the entity can have, \texttt{regen} which determines how much health should the entity gain on each regeneration
period of the health system, and \texttt{defense}, which determines the amount of damage that should be subtracted from incoming damage.

Once we define all of the components of our entity, we can register her in the list of all defined entities using the function
\texttt{game.entity.register}.

\begin{listing}
    \centering
    \begin{lstlisting}
-- Table defining the ogre entity.
ogre = {
    -- List of components the ogre entity has.
    components = {
        game.enum.component.physics,
        game.enum.component.health
    },

    -- Definition of the physics component.
    PhysicsComponent = {
        solid = false
    },

    -- Definition of the health component.
    HealthComponent = {
        max_hp = 1000,
        regen = 10,
        defense = 50
    }
}

game.entity.register("ogre")
    \end{lstlisting}
    \caption{Definition of a simple entity.}
    \label{ent-table-ex1}
\end{listing}

\subsection{Blueprints}

Components can also define the behavior of an entity. To implement this, the game contains the concept of \emph{blueprints}. Blueprint
is a table that contains functions that describe the behavior of an entity of a specific type. Which blueprint is used is defined
for each component individually, this allows us to combine the behavior of different entity types into a new one without the need
to create new blueprints.

In Listing~\ref{ent-table-ex2} we can see an implementation of an example healer entity. This entity has three components that describe
its behavior. \texttt{SpellComponent} describes the ability to cast spells and contains fields \texttt{blueprint}, which determines
which blueprint is used for spell casting, and \texttt{cooldown}, which determines the minimal time between two consecutive spell casts.
\texttt{AIComponent} describes the overall behavior the entity performs on every periodic AI update and contains the field
\texttt{blueprint}, which determines which blueprint is used for these AI updates. \texttt{OnHitComponent} describes how the entity
reacts when it gets attacked by an enemy and contains fields \texttt{blueprint}, which determines which blueprint is used to find the
reaction to enemy attacks, and \texttt{cooldown}, which determines the minimal time between two consecutive reactions -- this can be used
to prevent overflow of the game's log if the entity just notifies the player of the enemy attack.

\begin{listing}[H]
    \centering
    \begin{lstlisting}
cowardly_healer = {
    components = {
        game.enum.component.spell,
        game.enum.component.ai,
        game.enum.component.on_hit
    },

    SpellComponent = {
        blueprint = "healer_blueprint",
        cooldown = 25.0
    },

    AIComponent = {
        blueprint = "healer_blueprint"
    },

    OnHitComponent = {
        blueprint = "coward_blueprint",
        cooldown = 1.0
    }
}

game.entity.register("cowardly_healer")
    \end{lstlisting}
    \caption{An example of component definitions that use blueprints.}
    \label{ent-table-ex2}
\end{listing}

The entity behaves in the same way as a healer does when it casts a spell or has its AI updated, so these two components use the
\texttt{healer\_blueprint}, the implementation of which can be seen in Listing~\ref{ent-table-ex3}. This blueprint contains the
implementation of function used by all of the components our \texttt{cowardly\_healer} has, but the \texttt{on\_hit} function
is not used in this case because a regular healer entity heals itself and continues its normal behavior when it is attacked while this entity
is too cowardly to continue and simply runs away.

\begin{listing}[H]
    \centering
    \begin{lstlisting}
-- Functions describing the behavior of a healer.
healer_blueprint = {
    -- Associated with the spell component.
    cast = function(id)
        -- Heal all friends of entity <id> that are nearby.
    end,

    -- Associated with the ai component.
    update = function(id)
        -- If possible, heal friends. Otherwise, attack enemies.
    end,

    -- Associated with the on hit component.
    on_hit = function(id, enemy)
        -- Heal self.
    end
}
    \end{lstlisting}
    \caption{Implementation of the healer blueprint.}
    \label{ent-table-ex3}
\end{listing}

Because of this, our \texttt{cowardly\_healer} uses the \texttt{coward\_blueprint}, the implementation of which can be seen in
Listing~\ref{ent-table-ex4}, for its \texttt{OnHitComponent}. The \texttt{on\_hit} function of this blueprint then simply forces our
\texttt{cowardly\_healer} to run away from its attacker whenever it gets attacked.

Since the individual functions may differ in both name and parameters for the different components that use blueprints, this information
is included in the component description in the file \texttt{components.txt}, which is a part of the Attachment~B.

Note that the implementation of these behavior blueprints is not required. However, because of the way the game operates to support
blueprints, every of these behavioral function needs to be contained within a Lua table whose name is set in the component's
\texttt{blueprint} name and has to have the signature that is required when it is a part of a blueprint.

\begin{listing}[H]
    \centering
    \begin{lstlisting}
-- Functions describing the behavior of a coward.
coward_blueprint = {
    -- Associated with the on hit component.
    on_hit = function(id, enemy)
        -- Run away from entity <enemy>.
    end
}
    \end{lstlisting}
    \caption{Implementation of the coward blueprint.}
    \label{ent-table-ex4}
\end{listing}

\section{Enemy waves}

The \emph{wave table} is a Lua table that defines the composition of enemy waves and delays between them. Each wave table contains the
\texttt{init} function, which is called whenever a level that uses the table is created. Additionally, it contains a pair of functions
\texttt{wstart\_X} and \texttt{wend\_X} for each of its waves, where \texttt{X} starts at 0 and gets incremented for each wave.

An example of the \texttt{init} function of a wave table can be seen in Listing~\ref{wave-table-init}. In it, the wave table resets the wave
composition that was set during any previous wave sequence by calling the function \texttt{game.wave.clear\_entity\_blueprints}
\footnote{In this case, the term blueprint does not refer to the blueprints we have discussed in the previous section, but to an
entire entity definition.}. It then sets the number of waves with \texttt{game.wave.set\_wave\_count}, resets the current wave number with
\texttt{game.wave.set\_curr\_wave\_number} and changes the time before the first wave starts with \texttt{game.wave.set\_countdown}.

\begin{listing}[H]
    \centering
    \begin{lstlisting}
wave = {
    -- Initializes this wave table.
    init = function()
        game.wave.clear_entity_blueprints()
        game.wave.set_wave_count(2)
        game.wave.set_curr_wave_number(0)
        game.wave.set_countdown(300)
    end
}
    \end{lstlisting}
    \caption{An example of the intialization function in a wave table.}
    \label{wave-table-init}
\end{listing}

In Listing~\ref{wave-table-first}, we can see an implementation of the starting and ending function of a first wave within a wave table.
When the first wave starts, the function \texttt{wstart\_0} is called. The starting functions generally define the composition
of the wave that is starting using the function \texttt{game.wave.set\_entity\_total}, which tells the wave system how many entities
comprise this wave so that it knows when to end the wave, and the function \texttt{game.wave.add\_entity\_blueprint}, which creates a new
member of the wave.

The entities that are part of the wave spawn on specific nodes that were set during level generation. If there are more entities in a wave
than there are spawning nodes, the wave system only spawns enough entities to cover these nodes at a time. Before it spawns another
group of entities, it waits a specific time period, which can be set using \texttt{game.wave.set\_spawn\_cooldown}.

Once all entities that belong to the first wave are killed, the \texttt{wend\_0} function is called, which generally only changes the
time before the next wave. If this time is not changed, the previously set value is used.

\begin{listing}[H]
    \centering
    \begin{lstlisting}
wave = {
    -- Called when the first wave starts.
    wstart_0 = function()
        game.wave.set_entity_total(2)
        game.wave.add_entity_blueprint("ogre")
        game.wave.add_entity_blueprint("coward_healer")
        game.wave.set_spawn_cooldown(10.0)
    end,

    -- Called when the first wave ends.
    wend_0 = function()
        game.wave.set_countdown(180)
    end
}
    \end{lstlisting}
    \caption{An example of the first wave definition in a wave table.}
    \label{wave-table-first}
\end{listing}

Following waves are defined similarly, but note that if we do not call the function \texttt{game.wave.clear\_entity\_blueprints} between
waves, the entity blueprints of the previous wave are not deleted and will also be included in the next wave. This can be used to create
growing groups with each wave without the need to re-add the entities.

\begin{listing}[H]
    \centering
    \begin{lstlisting}
wave = {
    -- Called when the second wave starts.
    wstart_1 = function()
        game.wave.set_entity_total(4)
        game.wave.add_entity_blueprint("ogre")
        game.wave.add_entity_blueprint("coward_healer")
        game.wave.set_spawn_cooldown(10.0)
    end,

    -- Called when the second wave ends.
    wend_1 = function()
        game.print("All enemies defeated!")
        game.print("Now here's even more enemies!")
        game.wave.turn_endless_mode_on()
    end
}
    \end{lstlisting}
    \caption{An example of the second wave definition in a wave table.}
    \label{wave-table-second}
\end{listing}

In Listing~\ref{wave-table-second} we can se the definition of the second and, in this case, the last wave defined within a wave table.
Its \texttt{wend} function does not call the blueprint resetting function so the total number of entities is set to four, because the two
entities from the previous wave also spawn in this wave.
If we want the wave sequence to repeat its last indefinitely wave once all waves are finnished, we can use the function
\texttt{game.wave.turn\_endless\_mode\_on} in the last \texttt{wend} function.

\section{Research}

The research nodes in the game are organized to a grid of six rows and seven columns. The game controls these nodes by passing the number
of the row and of the column the node is located in into three functions. The first of these functions, \texttt{game.gui.research.get\_name}
is called once at the start of the game and returns the name of the node located at the position passed as its parameters. The second
function, \texttt{game.gui.research.get\_price}, returns the price of the unlock of the research node and the third of these functions,
\texttt{game.gui.research.unlock} is called when the player buys the research node and performs the unlock of the node's benefit to the
player.

The unlocking function does not impose any limitations on the characteristic of the node. It can unlock new spell, new minion, new building
or perform a one time action that gives the player a bonus of sorts. An example of the research implementation can be found in the
file \texttt{scripts/research.lua}, but the game does not have any requirements on the implementation besides the functionality of these
three functions.

Note that the \texttt{game.gui.research} table, which should contain these functions, is already predefined by the game and as such we should
not overwrite it when we write our research node.

\section{Spells}

Similarly to the definition of a research node, spells are defined by a table that contains three functions and is itself contained within
the table \texttt{game.spell.spells}. Note that, unlike the research table, this table is not created by the engine but is created in the
script \texttt{scripts/spells.lua}. This means, that if we create a modification of the game, we can simply add new spell definitions
to this predefined table, but if we create our own replacement of the \texttt{scripts/spells.lua}, we need to create this table ourselves.

When the player selects a spell, the engine calls the function \texttt{init}, located in the spell table. This function should change the
player into casting state, often done by setting the type of the currently cast spell. When the player casts a spell that has been
previously initialized, the engine calls the two remaining function that are in the spell table.

Firstly, it calls \texttt{pay\_mana},
which is supposed to subtract the mana cost from the player and to return \texttt{true} if the player can cast the spell. If this function
returns \texttt{false}, the spell casting is interrupted and the player is notified that he has insufficient mana.

Secondly, the engine calls the function \texttt{cast} if the spell casting was not interrupted during the call to \texttt{pay\_mana}. This
function performs the actual spell cast. When we create a new spell, we can allow the player to use it by calling the function
\texttt{game.spell.register\_spell}, which accepts the name of the spell table -- without the \texttt{game.spell.spells} prefix -- as its
parameter, either directly inside a script file or inside a research node. Note that, unlike the previous two functions which have no
parameters, this function has different parameters for different spell types.

\subsubsection{Targeted spells}

In Listing~\ref{spell-ex-targeted}, we can see an example of one of these spell types -- a targeted spell. This type of spell  affects a
single currently selected entity. In its \texttt{init} function, it changes the type of the current spell to \texttt{targeted} using
the function \texttt{game.spell.set\_type}. The \texttt{targeted} type, along with other spell types, is a variable stored in the
table called \texttt{game.enum.spell\_type}. These enum tables are used in a similar way to \cpp enums that are defined in the source file
\texttt{Enums.hpp} -- the fields of these tables hold integer values that correspond with the \cpp values.

The \texttt{pay\_mana} function subtracts mana from the player using the function \texttt{game.player.sub\_mana}. This API function returns
\texttt{true} if the player has enough mana and false otherwise so the returned value can be returned from \texttt{pay\_mana} as well.
Because the implementation of this function is almost always the same, we will not list it in any of the other spell examples, but it is
always required regardless of spell type.

The \texttt{cast} function of a targeted spells accept the identifier of the selected entity as its parameter. In this case, it teleports
the target to a random unobstructed place in the game world and then uses the function \texttt{game.spell.stop\_casting}. This causes the
spell to be interrupted once it has be cast once, if we want to allow consecutive casts of the same spell, we can simply avoid using this
API function and let the player to decide when to stop casting.

\begin{listing}[H]
    \centering
    \begin{lstlisting}
-- Teleports the target to a random location.
game.spell.spells.random_teleport = {
    -- Initialization, sets the spell type.
    init = function()
        game.spell.set_type(game.enum.spell_type.targeted)
    end,

    -- Subtracts mana from the player and returns true
    -- if the player had enough, returns false otherwise.
    pay_mana = function()
        return game.player.sub_mana(50)
    end,

    -- Applies the effect of the spell to the targeted
    -- entity, which has identifier <target>.
    cast = function(target)
        game.grid.place_at_random_free_node(target)
        game.spell.stop_casting()
    end
}
    \end{lstlisting}
    \caption{An example of a targeted spell.}
    \label{spell-ex-targeted}
\end{listing}

\subsubsection{Positional spells}

In Listing~\ref{spell-ex-positional}, we can see an example of a positional spell. This type of spells has an effect at a specific point in
the game world. The \texttt{cast} function accepts two dimensional coordinates as its parameters, which specify this point with no regard
to height. It uses the \texttt{game.command.reposition} function, which commands the minion that is closest to the specified position to
move to the position, and then interrupts the cast to avoid the accidental repositioning of multiple minions.

This type of spells can be used for commands similar to this one, to spawn groups of entities or a single entity. However, since the
\texttt{init} function only sets the type of the spell and nothing else, the player would not see the single entity that is supposed
to be spawned following the mouse cursor as is common in many games. For this, the spell type presented in the following section is
more suitable.

Note that since the spell cannot be seen during cast and is activated when the player clicks somewhere within the game world, it's advised
to interrupt casting inside the \texttt{cast} function to avoid accidental repeated casts.

\begin{listing}[H]
    \centering
    \begin{lstlisting}
-- Orders the minion that is closest to the area of
-- the spell cast to move to that area.
game.spell.spells.order_repositioning = {
    -- Initialization, sets the spell type.
    init = function()
        game.spell.set_type(game.enum.spell_type.positional)
    end,

    -- Applies the effect of the spell, accepts
    -- two dimensional coordinates in the game world
    -- which denote the place where the player cast
    -- the spell.
    cast = function(x, y)
        game.command.reposition(x, y)
        game.command.stop_casting()
    end
}
    \end{lstlisting}
    \caption{An example of a positional spell.}
    \label{spell-ex-positional}
\end{listing}

\subsubsection{Placing spells}

In Listing~\ref{spell-ex-placing}, we can see an example of a placing spell. This type of spells is used when we want the spell to place a
new entity into the game world. In the \texttt{init} function, we -- besides setting the spell type -- invoke the entity placer by using
\texttt{game.entity.place} which accepts the name of an entity defining table as its argument and creates a model of the entity that follows
the mouse cursor. From this point onward, the placement is
performed by the engine and the \texttt{cast} function is only called after the entity has been placed and is provided with the identifier
of the placed entity as its argument so it can manipulate with it.

In this case, we spawn our \texttt{coward\_healer} entity and lower its health. Note that since we did not call
\texttt{game.spell.stop\_casting}, the placement process will continue until canceled or the player runs out of mana, but unlike positional
spells, placing spells provide visual hint of the ongoing cast -- the model of the entity that follows the mouse cursor.

\begin{listing}
    \centering
    \begin{lstlisting}
game.spell.spells.spawn_damaged_coward_healer = {
    -- Initialization, sets the spell type.
    init = function()
        game.spell.set_type(game.enum.spell_type.placing)
        game.entity.place("coward_healer")
    end,

    -- Does NOT apply the effect of the spell,
    -- that has been done by the engine.
    -- This function only manipulates the placed
    -- entity.
    cast = function(id)
        game.health.sub(id, 100)
    end
}
    \end{lstlisting}
    \caption{An example of a placing spell.}
    \label{spell-ex-placing}
\end{listing}

\subsubsection{Global spells}

The last type of spells that we can define are global spells, the example of which can be seen in Listing~\ref{spell-ex-global}. These
spells serve as Lua functions that are executed when the player casts them. This particular spell increases the amount of gold that the
player has whenever it is cast using the \texttt{game.player.add\_gold} API function.

These spells are cast on every release of the left mouse button, so the player may accidentally cast these spells multiple times because
they offer no visual hint of the ongoing cast. Because of this, using \texttt{game.spell.stop\_casting} to interrupt the casting process
in the \texttt{cast} function is advised.

\begin{listing}
    \centering
    \begin{lstlisting}
    game.spell.spells.cheat = {
        -- Initialization, sets the spell type.
        init = function()
            game.spell.set_type(game.enum.spell_type.global)
        end,
        
        -- Applies the effect of the spell.
        cast = function()
            game.player.add_gold(10000)
            game.spell.stop_casting()
        end,
    }
    \end{lstlisting}
    \caption{An example of a global spell.}
    \label{spell-ex-global}
\end{listing}
