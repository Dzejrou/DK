\chapter{Scripter's Documentation}

% TODO
\begin{itemize}
    \item talk about the API, possibly mention some API reference text file
        that I should make as an attachement
\end{itemize}

\section{Initialization}

% TODO
\begin{itemize}
    \item talk about the config.lua \& init.lua duo
    \item explain different options in config
    \item explain how scripts packs are added to init
    \item talk abut the core.lua file of each script pack
        and what it should do (probably with an example)
    \item explain how to add a new script directory
    \item explain how to override some behaviour
    \item explain the new level callback and for what it can be used
        (+ the return value meaning)
\end{itemize}

\section{Entity Representation}

% TODO
\begin{itemize}
    \item explain the structure of a script representing
        an entity (with a small example)
\end{itemize}

\section{Blueprints}

% TODO
\begin{itemize}
    \item explain what is a blueprint, how to make one and
        how to use it
    \item explain why it's a table and not just a function
        and why the functions in different blueprint should
        have different names
\end{itemize}

\section{Research}

% TODO
\begin{itemize}
    \item explain how to add new unblocks and modify existing ones
\end{itemize}

\section{Spells}

% TODO
\begin{itemize}
    \item explain the structure of a spell table and how to make new spell
    \item mention again the spell types and the difference they make
        in the spell table function implementations
    \item mention that spells can be made during run time for testing
        purposes
\end{itemize}

\section{Creating a Mod}

% TODO
\begin{itemize}
    \item provide a simple tutorial on how to make a new mod
    \item possibly the tower defense one?
\end{itemize}
